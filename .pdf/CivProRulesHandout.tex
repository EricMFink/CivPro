\documentclass[12pt,letterpaper,twoside]{article}
\usepackage[hmarginratio=1:1,top=1in,bottom=.5in,left=.25in,right=.25in]{geometry}

\usepackage{setspace}
\usepackage{ragged2e}
\RaggedRight
\usepackage[none]{hyphenat}
\setlength{\parindent}{0em}
\setlength{\parskip}{1em}
\raggedbottom
\setlength{\emergencystretch}{3em} % prevent overfull lines
\usepackage{multicol}

% ==========================
% = Language & Date Format =
% ==========================

\usepackage[english]{babel}
\usepackage{datetime}
\newdateformat{mydate}{\THEDAY-\monthname-\THEYEAR}

% ==================
% = Colors & Fonts =
% ==================

\usepackage[table]{xcolor}
\definecolor{OffBlack}{HTML}{191919}
\definecolor{Maroon}{HTML}{800000}
\definecolor{TarHeelBlue}{HTML}{4b9cd3}
\definecolor{HopkinsBlue}{HTML}{002D72}
\definecolor{RacingGreen}{HTML}{004225}

\usepackage{fontspec}
\setmainfont{Athelas}[Mapping={tex-text}]
\setsansfont{Adelle Sans}[Mapping=tex-text]
\newfontface\titlefont{Adelle Sans}[Colour=Maroon]
\setmonofont{LFT Etica Mono}[Mapping=tex-text,Scale=0.8]

% =======================
% = Headers and Footers = 
% =======================

\usepackage{fancyhdr}
\pagestyle{fancy}
\fancyfoot[C]{}
\fancyfoot[R]{\tiny\texttt{Revised: \mydate\today}}
\fancyfoot[L]{}
\fancyhead[C]{\small{FRCP \& Statutes for Final Exam}}
\fancyhead[LE]{\small{\thepage}}
\fancyhead[LO]{\small{}}
\fancyhead[RE]{\small{}}
\fancyhead[RO]{\small{\thepage}}

% ============
% = Headings =
% ============

\usepackage[compact]{titlesec}
% \titlespacing*{<command>}{<left>}{<before-sep>}{<after-sep>}
\titlespacing*{\section}{0em}{1.5em}{.5em}
\titlespacing*{\subsection}{0em}{1em}{.2em}
\titlespacing*{\subsubsection}{0em}{0em}{0em}
\titlespacing*{\paragraph}{0em}{0em}{0em}
\titlespacing*{\subparagraph}{0em}{0em}{0em}

\setcounter{secnumdepth}{0}

\titleformat{\section}[hang]
  {\Large\rmfamily}
  {\thesection}
  {1em}
  {}
\titleformat{\subsection}[hang]
  {\large\rmfamily}
  {\thesubsection}
  {1em}
  {}
\titleformat{\subsubsection}[hang]
  {\normalsize\bfseries\rmfamily}
  {\thesubsubsection}
  {1em}
  {}
\titleformat{\paragraph}[hang]
  {\normalsize\bfseries\rmfamily}
  {\theparagraph}
  {1em}
  {}
\titleformat{\subparagraph}[hang]
  {\normalsize\itshape\rmfamily}
  {\thesubparagraph}
  {1em}
  {}

% =========
% = Lists =
% =========
\usepackage{enumitem}
\setlistdepth{6}

\setlist[itemize,1]{label=\textbullet}
\setlist[itemize,2]{label=\textopenbullet}
\setlist[itemize,3]{label=\textperiodcentered}
\setlist[itemize,4]{label=\textendash}
\setlist[itemize,5]{label=\diamond}
\setlist[itemize,6]{label=\textopenbullet}

\setlist{nosep}
\providecommand{\tightlist}{%
  \setlength{\itemsep}{0em}\setlength{\parskip}{0em}}

% =============
% = Footnotes = 
% =============

\usepackage[hang,flushmargin]{footmisc} 

%\renewcommand{\footnotesize}{\fontsize{9pt}{11pt}\selectfont} %use to change size of footnotes

% ==============
% = Hyperlinks =
% ==============

\usepackage[unicode=true]{hyperref}
\hypersetup{%
  colorlinks=true,
  allcolors=HopkinsBlue,
  breaklinks=true,
  pdfusetitle=true,
  pdfauthor={Professor Fink},
  pdftitle={FRCP \& Statutes for Final Exam},
    pdfsubject={Civil Procedure},
      pdfproducer=LateX via pandoc,
  pdfcreator=LateX via pandoc
}

% ===============================
% ===== BEGIN DOCUMENT ==========
% ===============================

\begin{document}

% Sets default font color without overriding hyperlink colors
\color{OffBlack}

\singlespacing
\thispagestyle{empty}
\begin{flushleft}

\footnotesize{Civil Procedure} \\
\footnotesize{Eric M. Fink} \\

\vspace{3mm}
\Large{FRCP \& Statutes for Final Exam} 


\setlength{\columnsep}{3em}
\begin{multicols*}{2}
\subsection{I. Federal Rules of Civil
Procedure}\label{i.-federal-rules-of-civil-procedure}

\subsubsection{Rule 8. General Rules of
Pleading}\label{rule-8.-general-rules-of-pleading}

\begin{enumerate}
\def\labelenumi{(\alph{enumi})}
\tightlist
\item
  Claim for Relief. A pleading that states a claim for relief must
  contain:

  \begin{enumerate}
  \def\labelenumii{(\arabic{enumii})}
  \tightlist
  \item
    a short and plain statement of the grounds for the court's
    jurisdiction, unless the court already has jurisdiction and the
    claim needs no new jurisdictional support;
  \item
    a short and plain statement of the claim showing that the pleader is
    entitled to relief; and
  \item
    a demand for the relief sought, which may include relief in the
    alternative or different types of relief.
  \end{enumerate}
\item
  Defenses; Admissions and Denials.

  \begin{enumerate}
  \def\labelenumii{(\arabic{enumii})}
  \tightlist
  \item
    In General. In responding to a pleading, a party must:

    \begin{enumerate}
    \def\labelenumiii{(\Alph{enumiii})}
    \tightlist
    \item
      state in short and plain terms its defenses to each claim asserted
      against it; and
    \item
      admit or deny the allegations asserted against it by an opposing
      party.
    \end{enumerate}
  \item
    Denials---Responding to the Substance. A denial must fairly respond
    to the substance of the allegation.
  \item
    General and Specific Denials. A party that intends in good faith to
    deny all the allegations of a pleading---including the
    jurisdictional grounds---may do so by a general denial. A party that
    does not intend to deny all the allegations must either specifically
    deny designated allegations or generally deny all except those
    specifically admitted.
  \item
    Denying Part of an Allegation. A party that intends in good faith to
    deny only part of an allegation must admit the part that is true and
    deny the rest.
  \item
    Lacking Knowledge or Information. A party that lacks knowledge or
    information sufficient to form a belief about the truth of an
    allegation must so state, and the statement has the effect of a
    denial.
  \item
    Effect of Failing to Deny. An allegation---other than one relating
    to the amount of damages---is admitted if a responsive pleading is
    required and the allegation is not denied. If a responsive pleading
    is not required, an allegation is considered denied or avoided.
  \end{enumerate}
\item
  Affirmative Defenses.

  \begin{enumerate}
  \def\labelenumii{(\arabic{enumii})}
  \tightlist
  \item
    In General. In responding to a pleading, a party must affirmatively
    state any avoidance or affirmative defense, including:

    \begin{itemize}
    \tightlist
    \item
      accord and satisfaction;
    \item
      arbitration and award;
    \item
      assumption of risk;
    \item
      duress;
    \item
      contributory negligence;
    \item
      estoppel;
    \item
      failure of consideration;
    \item
      fraud;
    \item
      illegality;
    \item
      injury by fellow servant;
    \item
      laches;
    \item
      license;
    \item
      payment;
    \item
      release;
    \item
      res judicata;
    \item
      statute of frauds;
    \item
      statute of limitations; and
    \item
      waiver.
    \end{itemize}
  \end{enumerate}
\end{enumerate}

\subsubsection{Rule 9. Pleading Special
Matters}\label{rule-9.-pleading-special-matters}

\begin{enumerate}
\def\labelenumi{(\alph{enumi})}
\setcounter{enumi}{1}
\tightlist
\item
  Fraud or Mistake; Conditions of Mind. In alleging fraud or mistake, a
  party must state with particularity the circumstances constituting
  fraud or mistake. Malice, intent, knowledge, and other conditions of a
  person's mind may be alleged generally.
\end{enumerate}

\subsubsection{Rule 11. Signing Pleadings, Motions, and Other Papers;
Representations to the Court;
Sanctions}\label{rule-11.-signing-pleadings-motions-and-other-papers-representations-to-the-court-sanctions}

\begin{enumerate}
\def\labelenumi{(\alph{enumi})}
\tightlist
\item
  Signature. Every pleading, written motion, and other paper must be
  signed by at least one attorney of record in the attorney's name---or
  by a party personally if the party is unrepresented. The paper must
  state the signer's address, e-mail address, and telephone number.
  Unless a Rule or statute specifically states otherwise, a pleading
  need not be verified or accompanied by an affidavit. The court must
  strike an unsigned paper unless the omission is promptly corrected
  after being called to the attorney's or party's attention.
\item
  Representations to the Court. By presenting to the court a pleading,
  written motion, or other paper---whether by signing, filing,
  submitting, or later advocating it---an attorney or unrepresented
  party certifies that to the best of the person's knowledge,
  information, and belief, formed after an inquiry reasonable under the
  circumstances:

  \begin{enumerate}
  \def\labelenumii{(\arabic{enumii})}
  \tightlist
  \item
    it is not being presented for any improper purpose, such as to
    harass, cause unnecessary delay, or needlessly increase the cost of
    litigation;
  \item
    the claims, defenses, and other legal contentions are warranted by
    existing law or by a nonfrivolous argument for extending, modifying,
    or reversing existing law or for establishing new law;
  \item
    the factual contentions have evidentiary support or, if specifically
    so identified, will likely have evidentiary support after a
    reasonable opportunity for further investigation or discovery; and
  \item
    the denials of factual contentions are warranted on the evidence or,
    if specifically so identified, are reasonably based on belief or a
    lack of information.
  \end{enumerate}
\end{enumerate}

\subsubsection{Rule 12. Defenses and Objections: When and How Presented;
Motion for Judgment on the Pleadings; Consolidating Motions; Waiving
Defenses; Pretrial
Hearing}\label{rule-12.-defenses-and-objections-when-and-how-presented-motion-for-judgment-on-the-pleadings-consolidating-motions-waiving-defenses-pretrial-hearing}

\begin{enumerate}
\def\labelenumi{(\alph{enumi})}
\setcounter{enumi}{1}
\tightlist
\item
  How to Present Defenses. Every defense to a claim for relief in any
  pleading must be asserted in the responsive pleading if one is
  required. But a party may assert the following defenses by motion:

  \begin{enumerate}
  \def\labelenumii{(\arabic{enumii})}
  \tightlist
  \item
    lack of subject-matter jurisdiction;
  \item
    lack of personal jurisdiction;
  \item
    improper venue;
  \item
    insufficient process;
  \item
    insufficient service of process;
  \item
    failure to state a claim upon which relief can be granted; and
  \item
    failure to join a party under Rule 19.
  \end{enumerate}
\end{enumerate}

\subsubsection{Rule 13. Counterclaim and
Crossclaim}\label{rule-13.-counterclaim-and-crossclaim}

\begin{enumerate}
\def\labelenumi{(\alph{enumi})}
\tightlist
\item
  Compulsory Counterclaim.

  \begin{enumerate}
  \def\labelenumii{(\arabic{enumii})}
  \tightlist
  \item
    In General. A pleading must state as a counterclaim any claim
    that---at the time of its service---the pleader has against an
    opposing party if the claim:

    \begin{enumerate}
    \def\labelenumiii{(\Alph{enumiii})}
    \tightlist
    \item
      arises out of the transaction or occurrence that is the subject
      matter of the opposing party's claim; and
    \item
      does not require adding another party over whom the court cannot
      acquire jurisdiction.
    \end{enumerate}
  \end{enumerate}
\item
  Permissive Counterclaim. A pleading may state as a counterclaim
  against an opposing party any claim that is not compulsory.
\item
  Crossclaim Against a Coparty. A pleading may state as a crossclaim any
  claim by one party against a coparty if the claim arises out of the
  transaction or occurrence that is the subject matter of the original
  action or of a counterclaim, or if the claim relates to any property
  that is the subject matter of the original action. The crossclaim may
  include a claim that the coparty is or may be liable to the
  cross-claimant for all or part of a claim asserted in the action
  against the cross-claimant.
\item
  Joining Additional Parties. Rules 19 and 20 govern the addition of a
  person as a party to a counterclaim or crossclaim.
\end{enumerate}

\subsubsection{Rule 14. Third-Party
Practice}\label{rule-14.-third-party-practice}

\begin{enumerate}
\def\labelenumi{(\alph{enumi})}
\tightlist
\item
  When a Defending Party May Bring in a Third Party.

  \begin{enumerate}
  \def\labelenumii{(\arabic{enumii})}
  \tightlist
  \item
    Timing of the Summons and Complaint. A defending party may, as
    third-party plaintiff, serve a summons and complaint on a nonparty
    who is or may be liable to it for all or part of the claim against
    it. But the third-party plaintiff must, by motion, obtain the
    court's leave if it files the third-party complaint more than 14
    days after serving its original answer.
  \item
    Third-Party Defendant's Claims and Defenses. The person served with
    the summons and third-party complaint---the ``third-party
    defendant'':

    \begin{enumerate}
    \def\labelenumiii{(\Alph{enumiii})}
    \tightlist
    \item
      must assert any defense against the third-party plaintiff's claim
      under Rule 12;
    \item
      must assert any counterclaim against the third-party plaintiff
      under Rule 13(a), and may assert any counterclaim against the
      third-party plaintiff under Rule 13(b) or any crossclaim against
      another third-party defendant under Rule 13(g);
    \item
      may assert against the plaintiff any defense that the third-party
      plaintiff has to the plaintiff's claim; and
    \item
      may also assert against the plaintiff any claim arising out of the
      transaction or occurrence that is the subject matter of the
      plaintiff's claim against the third-party plaintiff.
    \end{enumerate}
  \item
    Plaintiff's Claims Against a Third-Party Defendant. The plaintiff
    may assert against the third-party defendant any claim arising out
    of the transaction or occurrence that is the subject matter of the
    plaintiff's claim against the third-party plaintiff. The third-party
    defendant must then assert any defense under Rule 12 and any
    counterclaim under Rule 13(a), and may assert any counterclaim under
    Rule 13(b) or any crossclaim under Rule 13(g).
  \end{enumerate}
\end{enumerate}

\subsubsection{Rule 15. Amended and Supplemental
Pleadings}\label{rule-15.-amended-and-supplemental-pleadings}

\begin{enumerate}
\def\labelenumi{(\alph{enumi})}
\tightlist
\item
  Amendments Before Trial.

  \begin{enumerate}
  \def\labelenumii{(\arabic{enumii})}
  \tightlist
  \item
    Amending as a Matter of Course. A party may amend its pleading once
    as a matter of course within:

    \begin{enumerate}
    \def\labelenumiii{(\Alph{enumiii})}
    \tightlist
    \item
      21 days after serving it, or
    \item
      if the pleading is one to which a responsive pleading is required,
      21 days after service of a responsive pleading or 21 days after
      service of a motion under Rule 12(b), (e), or (f), whichever is
      earlier.
    \end{enumerate}
  \item
    Other Amendments. In all other cases, a party may amend its pleading
    only with the opposing party's written consent or the court's leave.
    The court should freely give leave when justice so requires.
  \end{enumerate}
\item
  {[}omitted{]}
\item
  Relation Back of Amendments.

  \begin{enumerate}
  \def\labelenumii{(\arabic{enumii})}
  \tightlist
  \item
    When an Amendment Relates Back. An amendment to a pleading relates
    back to the date of the original pleading when:

    \begin{enumerate}
    \def\labelenumiii{(\Alph{enumiii})}
    \tightlist
    \item
      the law that provides the applicable statute of limitations allows
      relation back;
    \item
      the amendment asserts a claim or defense that arose out of the
      conduct, transaction, or occurrence set out---or attempted to be
      set out---in the original pleading; or
    \item
      the amendment changes the party or the naming of the party against
      whom a claim is asserted, if Rule 15(c)(1)(B) is satisfied and if,
      within the period provided by Rule 4(m) for serving the summons
      and complaint, the party to be brought in by amendment:

      \begin{enumerate}
      \def\labelenumiv{(\roman{enumiv})}
      \tightlist
      \item
        received such notice of the action that it will not be
        prejudiced in defending on the merits; and
      \item
        knew or should have known that the action would have been
        brought against it, but for a mistake concerning the proper
        party's identity.
      \end{enumerate}
    \end{enumerate}
  \end{enumerate}
\end{enumerate}

\subsubsection{Rule 18. Joinder of
Claims}\label{rule-18.-joinder-of-claims}

\begin{enumerate}
\def\labelenumi{(\alph{enumi})}
\tightlist
\item
  In General. A party asserting a claim, counterclaim, crossclaim, or
  third-party claim may join, as independent or alternative claims, as
  many claims as it has against an opposing party.
\end{enumerate}

\subsubsection{Rule 19. Required Joinder of
Parties}\label{rule-19.-required-joinder-of-parties}

\begin{enumerate}
\def\labelenumi{(\alph{enumi})}
\tightlist
\item
  Persons Required to Be Joined if Feasible.

  \begin{enumerate}
  \def\labelenumii{(\arabic{enumii})}
  \tightlist
  \item
    Required Party. A person who is subject to service of process and
    whose joinder will not deprive the court of subject-matter
    jurisdiction must be joined as a party if:

    \begin{enumerate}
    \def\labelenumiii{(\Alph{enumiii})}
    \tightlist
    \item
      in that person's absence, the court cannot accord complete relief
      among existing parties; or
    \item
      that person claims an interest relating to the subject of the
      action and is so situated that disposing of the action in the
      person's absence may:

      \begin{enumerate}
      \def\labelenumiv{(\roman{enumiv})}
      \tightlist
      \item
        as a practical matter impair or impede the person's ability to
        protect the interest; or
      \item
        leave an existing party subject to a substantial risk of
        incurring double, multiple, or otherwise inconsistent
        obligations because of the interest.
      \end{enumerate}
    \end{enumerate}
  \end{enumerate}
\item
  When Joinder Is Not Feasible. If a person who is required to be joined
  if feasible cannot be joined, the court must determine whether, in
  equity and good conscience, the action should proceed among the
  existing parties or should be dismissed. The factors for the court to
  consider include:

  \begin{enumerate}
  \def\labelenumii{(\arabic{enumii})}
  \tightlist
  \item
    the extent to which a judgment rendered in the person's absence
    might prejudice that person or the existing parties;
  \item
    the extent to which any prejudice could be lessened or avoided by:

    \begin{enumerate}
    \def\labelenumiii{(\Alph{enumiii})}
    \tightlist
    \item
      protective provisions in the judgment;
    \item
      shaping the relief; or
    \item
      other measures;
    \end{enumerate}
  \item
    whether a judgment rendered in the person's absence would be
    adequate; and
  \item
    whether the plaintiff would have an adequate remedy if the action
    were dismissed for nonjoinder.
  \end{enumerate}
\end{enumerate}

\subsubsection{Rule 20. Permissive Joinder of
Parties}\label{rule-20.-permissive-joinder-of-parties}

\begin{enumerate}
\def\labelenumi{(\alph{enumi})}
\tightlist
\item
  Persons Who May Join or Be Joined.

  \begin{enumerate}
  \def\labelenumii{(\arabic{enumii})}
  \tightlist
  \item
    Plaintiffs. Persons may join in one action as plaintiffs if:

    \begin{enumerate}
    \def\labelenumiii{(\Alph{enumiii})}
    \tightlist
    \item
      they assert any right to relief jointly, severally, or in the
      alternative with respect to or arising out of the same
      transaction, occurrence, or series of transactions or occurrences;
      and
    \item
      any question of law or fact common to all plaintiffs will arise in
      the action.
    \end{enumerate}
  \item
    Defendants. Persons---as well as a vessel, cargo, or other property
    subject to admiralty process in rem---may be joined in one action as
    defendants if:

    \begin{enumerate}
    \def\labelenumiii{(\Alph{enumiii})}
    \tightlist
    \item
      any right to relief is asserted against them jointly, severally,
      or in the alternative with respect to or arising out of the same
      transaction, occurrence, or series of transactions or occurrences;
      and
    \item
      any question of law or fact common to all defendants will arise in
      the action.
    \end{enumerate}
  \end{enumerate}
\end{enumerate}

\subsubsection{Rule 56. Summary
Judgment}\label{rule-56.-summary-judgment}

\begin{enumerate}
\def\labelenumi{(\alph{enumi})}
\tightlist
\item
  Motion for Summary Judgment or Partial Summary Judgment. A party may
  move for summary judgment, identifying each claim or defense --- or
  the part of each claim or defense --- on which summary judgment is
  sought. The court shall grant summary judgment if the movant shows
  that there is no genuine dispute as to any material fact and the
  movant is entitled to judgment as a matter of law. The court should
  state on the record the reasons for granting or denying the motion.
\item
  {[}omitted{]}
\item
  Procedures.

  \begin{enumerate}
  \def\labelenumii{(\arabic{enumii})}
  \tightlist
  \item
    Supporting Factual Positions. A party asserting that a fact cannot
    be or is genuinely disputed must support the assertion by:

    \begin{enumerate}
    \def\labelenumiii{(\Alph{enumiii})}
    \tightlist
    \item
      citing to particular parts of materials in the record, including
      depositions, documents, electronically stored information,
      affidavits or declarations, stipulations (including those made for
      purposes of the motion only), admissions, interrogatory answers,
      or other materials; or
    \item
      showing that the materials cited do not establish the absence or
      presence of a genuine dispute, or that an adverse party cannot
      produce admissible evidence to support the fact.
    \end{enumerate}
  \item
    \emph{Objection That a Fact Is Not Supported by Admissible
    Evidence.} A party may object that the material cited to support or
    dispute a fact cannot be presented in a form that would be
    admissible in evidence.
  \item
    \emph{Materials Not Cited.} The court need consider only the cited
    materials, but it may consider other materials in the record.
  \item
    \emph{Affidavits or Declarations.} An affidavit or declaration used
    to support or oppose a motion must be made on personal knowledge,
    set out facts that would be admissible in evidence, and show that
    the affiant or declarant is competent to testify on the matters
    stated.
  \end{enumerate}
\item
  {[}omitted{]}
\item
  Failing to Properly Support or Address a Fact. If a party fails to
  properly support an assertion of fact or fails to properly address
  another party's assertion of fact as required by Rule 56(c), the court
  may:

  \begin{enumerate}
  \def\labelenumii{(\arabic{enumii})}
  \tightlist
  \item
    give an opportunity to properly support or address the fact;
  \item
    consider the fact undisputed for purposes of the motion;
  \item
    grant summary judgment if the motion and supporting materials ---
    including the facts considered undisputed --- show that the movant
    is entitled to it; or
  \item
    issue any other appropriate order.
  \end{enumerate}
\end{enumerate}

\newpage

\subsection{II. Federal Statutes}\label{ii.-federal-statutes}

\subsubsection{28 U.S.C. § 1331. Federal
question}\label{u.s.c.-1331.-federal-question}

The district courts shall have original jurisdiction of all civil
actions arising under the Constitution, laws, or treaties of the United
States.

\subsubsection{28 U.S.C. § 1332. Diversity of citizenship; amount in
controversy;
costs}\label{u.s.c.-1332.-diversity-of-citizenship-amount-in-controversy-costs}

\begin{enumerate}
\def\labelenumi{(\alph{enumi})}
\tightlist
\item
  The district courts shall have original jurisdiction of all civil
  actions where the matter in controversy exceeds the sum or value of
  \$75,000, exclusive of interest and costs, and is between---

  \begin{enumerate}
  \def\labelenumii{(\arabic{enumii})}
  \tightlist
  \item
    citizens of different States;
  \end{enumerate}
\item
  {[}Omitted{]}
\item
  For the purposes of this section and section 1441 of this title---

  \begin{enumerate}
  \def\labelenumii{(\arabic{enumii})}
  \tightlist
  \item
    a corporation shall be deemed to be a citizen of every State and
    foreign state by which it has been incorporated and of the State or
    foreign state where it has its principal place of business {[}
    \ldots{]}
  \item
    the legal representative of the estate of a decedent shall be deemed
    to be a citizen only of the same State as the decedent, and the
    legal representative of an infant or incompetent shall be deemed to
    be a citizen only of the same State as the infant or incompetent.
  \end{enumerate}
\end{enumerate}

\subsubsection{28 U.S.C. § 1367. Supplemental
jurisdiction}\label{u.s.c.-1367.-supplemental-jurisdiction}

\begin{enumerate}
\def\labelenumi{(\alph{enumi})}
\tightlist
\item
  Except as provided in subsections (b) and (c) or as expressly provided
  otherwise by Federal statute, in any civil action of which the
  district courts have original jurisdiction, the district courts shall
  have supplemental jurisdiction over all other claims that are so
  related to claims in the action within such original jurisdiction that
  they form part of the same case or controversy under Article III of
  the United States Constitution. Such supplemental jurisdiction shall
  include claims that involve the joinder or intervention of additional
  parties.
\item
  In any civil action of which the district courts have original
  jurisdiction founded solely on section 1332 of this title, the
  district courts shall not have supplemental jurisdiction under
  subsection (a) over claims by plaintiffs against persons made parties
  under Rule 14, 19, 20, or 24 of the Federal Rules of Civil Procedure,
  or over claims by persons proposed to be joined as plaintiffs under
  Rule 19 of such rules, or seeking to intervene as plaintiffs under
  Rule 24 of such rules, when exercising supplemental jurisdiction over
  such claims would be inconsistent with the jurisdictional requirements
  of section 1332.
\end{enumerate}

\subsubsection{28 U.S.C. § 1441. Removal of civil
actions}\label{u.s.c.-1441.-removal-of-civil-actions}

\begin{enumerate}
\def\labelenumi{(\alph{enumi})}
\tightlist
\item
  Generally.--- Except as otherwise expressly provided by Act of
  Congress, any civil action brought in a State court of which the
  district courts of the United States have original jurisdiction, may
  be removed by the defendant or the defendants, to the district court
  of the United States for the district and division embracing the place
  where such action is pending.
\item
  Removal Based on Diversity of Citizenship.---

  \begin{enumerate}
  \def\labelenumii{(\arabic{enumii})}
  \tightlist
  \item
    In determining whether a civil action is removable on the basis of
    the jurisdiction under section 1332(a) of this title, the
    citizenship of defendants sued under fictitious names shall be
    disregarded.
  \item
    A civil action otherwise removable solely on the basis of the
    jurisdiction under section 1332(a) of this title may not be removed
    if any of the parties in interest properly joined and served as
    defendants is a citizen of the State in which such action is
    brought.
  \end{enumerate}
\end{enumerate}
\end{multicols*}

\end{flushleft}

\end{document}
