\documentclass[11pt,letterpaper,twoside]{article}
\usepackage[hmarginratio=1:1,top=1in,bottom=.75in,left=1.5in,right=1.5in]{geometry}

\usepackage{etoolbox}
\usepackage[]{ccicons}
\usepackage{fontawesome5}
\usepackage{marginnote}
\usepackage{sidenotes}

\usepackage{needspace}
\usepackage[defaultlines=3,all]{nowidow}
\raggedbottom

\usepackage{ragged2e}
\setlength{\RaggedRightParindent}{0em}
\RaggedRight

\usepackage[singlespacing]{setspace}

\usepackage[none]{hyphenat}

\usepackage{parskip}
\setlength{\parindent}{0em}
\setlength{\parskip}{0.6em plus 0.2em minus 0.1em}
\setlength{\emergencystretch}{3em}  % prevent overfull lines

% ==========================
% = Language & Date Format =
% ==========================

\usepackage[english]{babel}
\usepackage{currfile}

% ==================
% = Colors & Fonts =
% ==================

\usepackage[table]{xcolor}
\definecolor{OffBlack}{HTML}{191919}
\definecolor{Maroon}{HTML}{800000}
\definecolor{TarHeelBlue}{HTML}{4b9cd3}
\definecolor{HopkinsBlue}{HTML}{002D72}
\definecolor{RacingGreen}{HTML}{004225}

\usepackage{fontspec}
\setmainfont{Athelas}[Mapping=tex-text]
\setsansfont{Adelle Sans}[Mapping=tex-text]
\setmonofont[Scale=0.90]{LFT Etica Mono}

% font style for margin notes
%\renewcommand{\marginfont}{\color{RacingGreen}\sffamily\scriptsize}

% smaller sizes for footnotes & captions
%\renewcommand{\footnotesize}{\scriptsize}
%\renewcommand{\captionsize}{\small\itshape}

% =======================
% = Headers and Footers = 
% =======================

\usepackage{fancyhdr}
\pagestyle{fancy}
\fancyfoot{}
\fancyfoot[C]{}
\fancyfoot[R]{\tiny\texttt{ Revised: \today}}
\fancyfoot[L]{}
\fancyhead[CE]{\textsc{Civil Procedure}}
\fancyhead[CO]{\textsc{Elon University School of Law}}
\fancyhead[LE]{\textsc{\thepage}}
\fancyhead[LO]{\textsc{Fall 2025}}
\fancyhead[RE]{\textsc{Prof. Fink}}
\fancyhead[RO]{\textsc{\thepage}}

% =========
% = Lists =
% =========

\usepackage{enumitem}
\setlistdepth{6}
\setlist[itemize]{leftmargin={1.5em}}
\providecommand{\tightlist}{%
\setlength{\itemsep}{0em}\setlength{\parskip}{0em}\setlength{\itemindent}{0em}}

% ============
% = Headings =
% ============

\usepackage[compact]{titlesec}
% \titlespacing*{<command>}{<left>}{<before-sep>}{<after-sep>}
\titlespacing*{\section}{0em}{0em}{.5em}
\titlespacing*{\subsection}{0em}{2em}{.5em}
\titlespacing*{\subsubsection}{0em}{1.5em}{.5em}
\titlespacing*{\paragraph}{0em}{1em}{0em}
\titlespacing*{\subparagraph}{0em}{.2em}{0em}

\setcounter{secnumdepth}{0}

% H1
\titleformat{\section}[block]
  {\LARGE\sffamily\color{Maroon}}
  {}
  {0em}
  {}

% H2
\titleformat{\subsection}[block]
  {\needspace{9\baselineskip}\Large\sffamily\color{Maroon}}
  {}
  {0em}
  {}
% H3
\titleformat{\subsubsection}[block]
  {\needspace{6\baselineskip}\large\rmfamily\bfseries}
  {\thesubsection}
  {0em}
  {}
% H4
\titleformat{\paragraph}[block]
  {\footnotesize\sffamily}
  {\thesubsubsection}
  {0em}
  {}
% H5
\titleformat{\subparagraph}[block]
  {\footnotesize\rmfamily}
  {}
  {0em}
  {}

% ==========
% = Quotes =
% ==========

\usepackage{csquotes}

\AtBeginEnvironment{quote}{\footnotesize\itshape} 

\renewenvironment{quote}{%
   \list{}{%
     \leftmargin2em   % this is the adjusting screw
     \rightmargin\leftmargin\parsep .1em }
   \item\relax
}
{\endlist}

% ==========
% = Tables =
% ==========

\usepackage{longtable,booktabs,array}
%\setlength\heavyrulewidth{0pt}
\usepackage{multirow}
\usepackage{calc} % for calculating minipage widths
% Correct order of tables after \paragraph or \subparagraph
\makeatletter
\patchcmd\longtable{\par}{\if@noskipsec\mbox{}\fi\par}{}{}
\makeatother
% Allow footnotes in longtable head/foot
\IfFileExists{footnotehyper.sty}{\usepackage{footnotehyper}}{\usepackage{footnote}}
\makesavenoteenv{longtable}
% set table font size & space between rows
\renewcommand{\arraystretch}{1.5}
\AtBeginEnvironment{longtable}{\footnotesize}

% =============
% = Footnotes = 
% =============

\usepackage[hang,flushmargin]{footmisc} 

%\renewcommand{\footnotesize}{\fontsize{9pt}{11pt}\selectfont} %use to change size of footnotes

% ==============
% = Hyperlinks =
% ==============

\usepackage[unicode=true]{hyperref}
\hypersetup{%
  colorlinks=true,
  allcolors=HopkinsBlue,
  breaklinks=true,
  pdfusetitle=true,
  pdfauthor={Eric M. Fink},
  pdftitle={Syllabus},
    pdfsubject={Civil Procedure},
      pdfproducer=LateX via pandoc,
  pdfcreator=LateX via pandoc
}

% ===============================
% ===== BEGIN DOCUMENT ==========
% ===============================

\begin{document}\thispagestyle{empty}

% Sets default font color without overriding hyperlink colors
\color{OffBlack}

% \maketitle

\section{Civil Procedure}

\begin{footnotesize}
\subparagraph{Elon University School of Law}
\subparagraph{Fall 2025}
\subparagraph{Room 207}
\subparagraph{Mondays, Wednesdays, \& Fridays, 1:30--3:15 pm}
\subparagraph{\url{emfink.net/CivPro}}
\vspace{1em}

\paragraph{Professor}
\subparagraph{Eric M. Fink} 
\subparagraph{efink@elon.edu}
\subparagraph{336.279.9334} 
\subparagraph{Office Hours: {\url{calendly.com/emfink/}}}

\vspace{1em}

\paragraph{Teaching Assistants:}
\subparagraph{Davi Thornton}
\subparagraph{dthornton2@elon.edu}
\vspace{.5em}
\subparagraph{Kristian Ellis}
\subparagraph{kellis12@elon.edu}
\end{footnotesize}

\vspace{1em}

\needspace{1\baselineskip}

\subsection{Description}\label{description}

In this course, you will learn about the procedures for civil suits.
Topics to be covered include the scope of a lawsuit, selection of an
appropriate forum, presentation of claims and defenses, choice of
applicable law, disposition without a trial, and the effect of judgments
on future litigation. Other aspects of civil litigation (e.g.,
discovery, trials, \& appeals) are covered in upper-level elective
courses. While the course will focus on federal courts, the rules of
civil procedure in many states (including North Carolina) are similar.

This course is intended to prepare you for legal practice by developing
the knowledge and skill required to recognize and analyze procedural
issues in civil litigation, advise clients on those issues and the
available options, and draft pleadings and motions.

\subsection{Course Materials}\label{course-materials}

\subsubsection{Required}\label{required}

Civil Procedure: An Open-Source Casebook (v.4.1 2025). Available on the
course website:
\href{https://www.emfink.net/CivPro/}{emfink.net/CivPro}.

\href{https://clickandlearnguide.com}{Click \& Learn Civil Procedure}
(Carolina Academic Press). Class Code: 325-138-0701.

\subsubsection{Suggested}\label{suggested}

Kevin M. Clermont, \emph{Principles of Civil Procedure} (West Academic
7th ed.~2024). Digital version available at no cost through the
\href{https://subscription.westacademic.com/}{West Academic Online Study
Aids Collection}. This hornbook provides concise overviews of the topics
covered in the course.

Joseph W. Glannon, \emph{Examples \& Explanations for Civil Procedure}
(Aspen 9th ed.) or \emph{The Glannon Guide to Civil Procedure} (Aspen
5th ed.). These are two popular study aids for Civil Procedure,
presenting similar content in different formats.

\subsection{Policies}\label{policies}

\subsubsection{Grading}\label{grading}

Your final grade for the term will be based on the assigned Click \&
Learn lessons (20\%) and a final exam (80\%):

\begin{itemize}
\tightlist
\item
  The Click \& Learn assignments are intended to help you review \&
  assess your understanding of the topics covered. The Schedule \&
  Assignments section below lists the corresponding Click \& Learn unit
  for each part of the course. Required lessons in each unit are
  indicated by a due date of December 1.\footnote{Click \& Learn lessons
    without a due date are not required and will not count toward your
    final grade.} You may work on the Click \& Learn lessons at your own
  pace and may redo them as many times as you like, as long as you
  complete them all by December 1. Your grade will be based on your
  highest score for each lesson.
\item
  The final exam (closed-book/closed-note) will consist of essay and
  short-answer questions. You will take the final exam at the Law School
  during the Fall Term exam period (date \& time TBA). Sample questions
  from past exams are posted on the
  \href{https://www.emfink.net/CivPro/}{course website}.
\end{itemize}

\subsubsection{Attendance}\label{attendance}

Elon Law School has adopted the following attendance policy for all
courses:

\begin{quote}
The Law School administers a policy that a student maintain regular and
punctual class attendance in all courses in which the student is
registered, including externships, clinical courses, or simulation
courses. Faculty members will give students written notice of their
attendance policies before or during the first week of class. These
policies may include, but are not limited to: treating late arrivals,
early departures, and/or lack of preparation as absences imposing grade
or point reductions for absences, including assigning a failing grade or
involuntarily withdrawing a student from the class and any other
policies that a professor deems appropriate to create a rigorous and
professional classroom environment.

In case of illness or emergency, students may contact the Office of
Student and Professional Life, which will then notify the student's
instructors. A student may notify the faculty member directly of a
planned absence and should refer to individual faculty members regarding
any policy that may apply. In the case of prolonged illness or
incapacity, the student should contact the
\href{https://www.elon.edu/u/law/students/}{Office of Student Life}.
\end{quote}

You should let me know (in advance if feasible) if you are unable to
attend class, will arrive late, or must leave early. I do not require an
explanation of the reason, nor do I require a doctor's note or other
documentation.

\subsubsection{Disability
Accommodations}\label{disability-accommodations}

For disability accommodation requests, contact the
\href{https://www.elon.edu/u/academics/koenigsberger-learning-center/disabilities-resources/homepage/graduate-student-resources/}{Elon
Disability Services Office}.

\subsubsection{Honor Code}\label{honor-code}

The Law School
\href{https://www.elon.edu/u/law/students/honor-code/}{honor code}
applies to all activities related to your law school study, including
conduct during class and examinations.

\subsection{Schedule \& Assignments}\label{schedule-assignments}

(\emph{Note:} This class will not meet on Sept.~24 and Oct.~10)

\subsubsection{1. Foundations of Civil
Procedure}\label{foundations-of-civil-procedure}

\emph{Casebook:} chap.~1; \emph{Clermont:} chap.~1 \& 2\footnote{Relevant
  portions of the Clermont hornbook are listed for each unit. These are
  optional, but you may find them helpful to provide context for the
  assigned reading.}

\emph{Click \& Learn}: Basic Skills \& Fundamentals

\textbf{Sept.~3:} At the Threshold; Civil Litigation in Federal Court,
\emph{Casebook} §§ 1.1 \& 1.2

\subsubsection{2. Parties and Claims}\label{parties-and-claims}

\emph{Casebook:} chap.~2; \emph{Clermont:} chap.~6

\emph{Click \& Learn}: Unit 6

\textbf{Sept.~5 \& 8:} Permissive Joinder of Claims \& Parties,
\emph{Casebook}: §§ 2.1 \& 2.2

\textbf{Sept.~10 \& 12:} Counterclaims, \emph{Casebook}: § 2.3

\textbf{Sept.~15:} Crossclaims, \emph{Casebook}: § 2.4

\textbf{Sept.~17 \& 19:} Third-Party Defendants \& Claims,
\emph{Casebook}: § 2.5

\textbf{Sept.~22:} Required Parties, \emph{Casebook}: chap.~2: § 2.6

\subsubsection{3. Personal Jurisdiction}\label{personal-jurisdiction}

\emph{Casebook:} chap.~3; \emph{Clermont:} chap.~4: §§ 4.2-4.4

\emph{Click \& Learn}: Unit 1

\textbf{Sept.~26:} Due Process and State Power, \emph{Casebook}: § 3.1

\textbf{Sept.~29 \& Oct.~1:} Specific Jurisdiction, \emph{Casebook}: §
3.2

\textbf{Oct.~3:} General Jurisdiction, Consent, \& Long-Arm Statutes,
\emph{Casebook}: §§ 3.3-3.5

\subsubsection{4. Subject Matter
Jurisdiction}\label{subject-matter-jurisdiction}

\emph{Casebook:} chap.~4; \emph{Clermont:} chap.~4: § 4.1

\emph{Click \& Learn}: Unit 3

\textbf{Oct.~6:} Federal Question Jurisdiction, \emph{Casebook}: § 4.1

\textbf{Oct.~8 \& 13:} Diversity Jurisdiction, \emph{Casebook}: § 4.2

\textbf{Oct.~15 \& 17:} Supplemental Jurisdiction, \emph{Casebook}: §
4.3

\textbf{Oct.~20:} Removal, \emph{Casebook}: § 4.4

\subsubsection{5. Choice of Governing
Law}\label{choice-of-governing-law}

\emph{Casebook:} chap.~5; \emph{Clermont:} chap.~3

\emph{Click \& Learn}: Unit 9

\textbf{Oct.~22 \& 24:} Rules of Decision Act \& Erie, \emph{Casebook}:
§ 5.1

\textbf{Oct.~27 \& 29:} Rules Enabling Act \& Hanna, \emph{Casebook}: §
5.2 \& 5.3

\subsubsection{6. Pleading}\label{pleading}

\emph{Casebook:} chap.~6; \emph{Clermont:} chap.~2: § 2.2(A)

\emph{Click \& Learn}: Unit 5

\textbf{Oct.~31}: Pleading Under the FRCP, \emph{Casebook}: §§ 6.1

\textbf{Nov.~3 \& 5}: Claims for Relief, \emph{Casebook}: §§ 6.2

\textbf{Nov.~7:} Responsive Pleading, \emph{Casebook}: § 6.3

\textbf{Nov.~10:} Amended Pleadings, \emph{Casebook}: § 6.4

\textbf{Nov.~12:} Truthfulness and Good Faith, \emph{Casebook}: § 6.5

\subsubsection{7. Preclusion}\label{preclusion}

\emph{Casebook:} chap.~7; \emph{Clermont:} chap.~5

\emph{Click \& Learn}: Unit 10 (part 1 only)

\textbf{Nov.~14 \& 17:} General Principles \& Claim Preclusion,
\emph{Casebook}: §§ 7.1 \& 7.2

\textbf{Nov.~19 \& 21:} Issue Preclusion, \emph{Casebook}: § 7.3

\subsubsection{8. Summary Judgment}\label{summary-judgment}

\emph{Casebook:} chap.~8; \emph{Clermont:} chap.~2: § 2.2

\emph{Click \& Learn}: Unit 8 (part 1: sec.~III only)

\textbf{Nov.~24:} Standard \& Burden of Production, \emph{Casebook}: §§
8.1 \& 8.2

\begin{center}\rule{0.5\linewidth}{0.5pt}\end{center}

\textbf{TBA:} Final Review

\textbf{Dec.~8:} Final Exam

\end{document}
